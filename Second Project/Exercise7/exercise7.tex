\documentclass[Second Project.tex]{subfiles}

\begin{document}
\section{ Άσκηση 7 }
Στην \textbf{7η άσκηση} ζητείται να εκτιμηθεί η τιμή κλεισίματος ημέρας δύο διαφορετικών μετοχών εταιρειών του
Χρηματιστηρίου Αθηνών για την ημέρα που είναι κοντινότερη στις \textbf{17 Μαρτίου 2020}. Οι εταιρείες που 
επιλέχθηκαν είναι η \textlatin{\textbf{AUTOHELLAS A.E.}} \textbf{(ΟΤΟΕΛ)} και τα 
\textbf{Ελληνικά Πετρέλαια (ΕΛΠΕ)}. Η εκτίμηση θα γίνει για την τιμή κλεισίματος αυτών των μετοχών στις
18 και 24 Μαρτίου 2020 χρησιμοποιοώντας 10 τιμές κλείσιματος από τις συνεδριάσεις που έγιναν από τις 4 έως και τις
17 Μαρτίου 2020, ενώ τα μοντέλα που θα χρησιμοποιηθούν είναι πολυώνυμα 2ου,3ου και 4ου βαθμού χρησιμοποιώντας
την μέθοδο των ελαχίστων τετραγώνων. Για την αξιολόγηση των μοντέλων θα χρησιμοποιηθεί η μετρική 
\textlatin{\textbf{Root Mean Squared Error - RMSE}}, ενώ εκτός από τις 10 τιμές για την εκπαίδευση των μοντέλων
θα χρησιμοποιηθούν και επιπλέον 5 τιμές για την αξιολόγηση των μοντέλων. Έτσι, συνολικά έχουμε 15 τιμές 
κλεισίματος από τις οποίες οι 10 (\textbf{$\approx$ 70\%} του συνολικού συνόλου δεδομένων) θα χρησιμοποιηθούν για την 
εκπαίδευση \textlatin{\textbf{(train set)}}, ενώ οι 5 (\textbf{$\approx$ 30\%} του συνολικού συνόλου δεδομένων) θα 
χρησιμοποιηθούν για την αξιολόγηση των μοντέλων \textlatin{\textbf{(test set)}}. Τέλος, τονίζεται ότι στα 
διαγράμματα αυτής της άσκησης η αριθμήση ξεκινάει στους άξονες από τις 4 Μαρτίου 2020, δηλαδή το $x=0$ 
αναφέρεται στις 4 Μαρτίου 2020 ενώ στην συνέχεια κάθε αύξηση του $x$ κατά 1 αναφέρεται στην επόμενη μέρα που
είναι ανοιχτό το χρηματιστήριο.
\end{document}