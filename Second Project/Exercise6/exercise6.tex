\documentclass[Second Project.tex]{subfiles}

\begin{document}
\section{ Άσκηση 6 }
Στην \textbf{6η άσκηση} ζητείται να υπολογιστεί το ολοκλήρωμα της συνάρτησης του ημιτόνου στο 
διάστημα $[0,\pi/2]$ χρησιμοποιώντας 11 σημεία με τις μεθόδους \textlatin{\textbf{Simpson}} και
\textbf{τραπεζίου}. Ζητείται δηλαδή η τιμή του 
\begin{equation*}
    \int_{0}^{\frac{\pi}{2}} sin(x) \,dx ,
\end{equation*}

καθώς και να αναφερθεί το θεωρητικό και αριθμητικό σφάλμα για κάθε μέθοδο αντίστοιχα. Τα 11 σημεία που
επιλέχθηκαν είναι τα παρακάτω
\begin{center}
    \begin{tabular}{ |c| } 
     \hline
     0.0 \\
     \hline
     0.15707963 \\
     \hline
     0.31415927 \\ 
     \hline
     0.4712389 \\
     \hline
     0.62831853 \\
     \hline
     0.78539816 \\
     \hline
     0.9424778 \\
     \hline
     1.09955743 \\
     \hline
     1.25663706 \\
     \hline
     1.41371669 \\
     \hline
     1.57079633 \\
     \hline
    \end{tabular}
\end{center}

Αν υπολογίσουμε το ολοκλήρωμα χρησιμοποιώντας γνώσεις του ολοκληρωτικού λογισμού έχουμε διαδοχικά
\begin{equation*}
    \int_{0}^{\frac{\pi}{2}} sin(x) \,dx = -cosx \Big|_0^\frac{\pi}{2} = 1
\end{equation*}
\end{document}