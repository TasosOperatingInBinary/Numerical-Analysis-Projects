\documentclass[Second Project.tex]{subfiles}

\begin{document}

\section{ Άσκηση 5 }
Στην \textbf{5η άσκηση} ζητείται να προγραμματιστεί μία συνάρτηση που να υπολογίζει το ημίτονο μιας οποιασδήποτε γωνίας.
Για να δημιουργηθεί αυτή η συνάρτηση θα χρησιμοποιηθούν 10 τιμές του ημιτόνου με πλήρη ακρίβεια που υποστηρίζει το μηχάνημα που τρέχει ο κώδικας
για την ανεξάρτητη μεταβλητή και 10 ψηφία ακρίβεια για την εξαρτημένη μεταβλητή. Η συνάρτηση του ημιτόνου θα προσεγγιστεί με τις παρακάτω 
μεθόδους :
\begin{itemize}
    \item Με πολυωνυμική προσέγγιση
    \item Με την μέθοδο \textlatin{Splines}
    \item Με την μέθοδο ελαχίστων τετραγώνων
\end{itemize}

Τα σημεία που θα χρησιμοποιηθούν είναι τα παρακάτω: 
\begin{center}
    \begin{tabular}{ |c|c| } 
    \hline
    \textlatin{x} & $sinx$ \\ \hline
    0.0 & 0.0 \\ \hline
    0.65 & 0.6051864057 \\ \hline
    1.3 & 0.9635581854 \\  \hline
    1.9500000000000002 & 0.9289597150 \\ \hline
    2.6 & 0.5155013718 \\  \hline
    3.25 & -0.1081951345 \\ \hline
    3.9000000000000004 & -0.6877661591 \\ \hline
    4.55 & -0.9868438585 \\ \hline
    5.2 & -0.8834546557 \\ \hline
    5.8500000000000005 & -0.4197640178 \\
    \hline
    \end{tabular}
\end{center}

Τα σημεία επιλέχθηκαν στο διάστημα $[0,2\pi]$ καθώς το ημίτονο είναι περιοδική συνάρτηση με περίοδο $Τ = 2\pi$ και σε 
διάστημα μήκους ίσο με την περίοδο η συνάρτηση περιέχει όλη την πληροφορία που χρειάζεται.
Στην συνέχεια γίνεται σύγκριση των παραπάνω προσεγγίσεων στο διάστημα $[-\pi , \pi]$ όσον αφορά την ακρίβεια 
προσέγγισης και προβάλλεται σε διάγραμμα το σφάλμα προσέγγισης για 200 σημεία στο διάστημα $[-\pi , \pi]$ και 
αναφέρεται ο αριθμός των ψηφίων ακρίβειας που πετυχαίνονται.
\end{document}