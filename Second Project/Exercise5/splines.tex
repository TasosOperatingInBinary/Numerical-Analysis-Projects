\documentclass[Second Project.tex]{subfiles}

\begin{document}

\subsection{ Προσέγγιση με \textlatin{Splines} }

Για την προσέγγιση με την μέθοδο \textlatin{Splines} χρησιμοποιούνται φυσικές κυβικές \textlatin{Splines}. Αρχικά, 
στην γενικότερη μορφή η μέθοδος \textlatin{Splines} προσεγγίζει με μία καμπύλη ( ευθεία, πολυώνυμο 2ου ή 3ου βαθμού),
ανά δύο τα σημεία εκπαίδευσης, δηλαδή χρησιμοποιεί διαφορετικούς τύπους για να προσεγγίσει αυτά τα σημεία εκπαίδευσης.
Ο όρος κυβικές που χρησιμοποιήθηκε παραπάνω αναφέρεται στο ότι η καμπύλη που θα προσεγγίσει ανά δύο τα σημεία εκπαίδευσης
είναι 3ου βαθμού. Επομένως, αφού έχουμε ανάφερει τα παραπάνω καταλήγουμε στον παρακάτω ορισμό για την κυβική 
\textlatin{Spline} για $n$ σημεία εκπαίδευσης
\begin{gather*}
    S_{1}(x) = y_{1} +b_{1}(x-x_{1}) + c_{1}(x-x_{1})^{2} + d_{1}(x-x_{1})^{3} , \hspace{5px} x_{1} \leq x \leq x_{2} \\
    S_{2}(x) = y_{2} +b_{2}(x-x_{2}) + c_{2}(x-x_{2})^{2} + d_{2}(x-x_{2})^{3} , \hspace{5px} x_{2} \leq x \leq x_{3} \\ 
    \vdots \\
    S_{n-1}(x) = y_{n-1} +b_{n-1}(x-x_{n-1}) + c_{n-1}(x-x_{n-1})^{2} + d_{n-1}(x-x_{n-1})^{3} , \hspace{5px} x_{n-1} \leq x \leq x_{n} \\
\end{gather*}
Με τις εξής ιδιότητες 
\begin{itemize}
    \item $S_{i}(x_{i}) = y_{i} $ και $S_{i}(x_{i+1}) = y_{i+1} $ \hspace{5px} για \hspace{5px}$i = 1, \dots , n-1$
    \item $ S'_{i-1}(x_{i}) = S'_{i}(x_{i}) $ \hspace{5px} για \hspace{5px}$i = 2, \dots, n-1$
    \item $S''_{i-1}(x_{i}) = S''_{i}(x_{i}) $ \hspace{5px} για \hspace{5px}$i = 2, \dots, n-1$
\end{itemize}
Χρησιμοποιώντας τα παραπάνω λαμβάνουμε $3n-5$ εξισώσεις ενώ έχουμε $3n-3$ αγνώστους. Επομένως, υπάρχουν άπειρες κυβικές \textlatin{Splines}
που προσεγγίζουν τα $n$ σημεία εκπαίδευσης. Εδώ εισέρχεται ο όρος φυσικές κυβικές \textlatin{Splines} που εισάγει τις δύο παρακάτω εξισώσεις
\begin{gather*}
    S''_{1}(x_{1}) = 0 \\
    S''_{n-1}(x_{n}) = 0
\end{gather*}

και επομένως πλέον έχουμε $3n-3$ εξισώσεις με $3n-3$ αγνώστους. Για να λύσουμε το σύστημα αρχικά ορίζουμε ως $c_{n} = S''_{n-1}(x_{n})/2$,
$\delta_{i} = x_{i+1} - x_{i}$ και $ \Delta_{i} = y_{i+1}-y_{i}$. Τελικά, με τους παραπάνω συμβολισμούς οι συντελεστές υπολογίζονται ως εξής
\begin{gather*}
    d_{i} = \frac{c_{i+1}-c{i}}{3\delta_{i}} \\
    b_{i} = \frac{\Delta_{i}}{\delta_{i}} - \frac{\delta_{i}}{3}(2c_{i} + c_{i+1})
\end{gather*}
για $i = 1, \dots, n-1$ και χρησιμοποιώντας τις εξισώσεις των κυβικών \textlatin{Splines} έχουμε ότι $2c_{1}=0$ και $2c_{n}=0$. Οπότε,
καταλήγουμε στο παρακάτω σύστημα προς επίλυση \\

\vspace{10mm}
$
\begin{pmatrix}
    1 & 0 & 0 &  &  & \\
    \delta_{1} & 2\delta_{1} + 2\delta_{2} & \delta_{2} & \ddots & & \\
    0 & \delta_{2} & 2\delta_{2} + 2\delta_{3} & \delta_{3} &  & \\ 
     & \ddots & \ddots & \ddots & \ddots & \\
     & &  & \delta_{n-2} & 2\delta_{n-2} + 2\delta_{n-1} & \delta_{n-1}\\
     & &  & 0 & 0 & 1\\
\end{pmatrix}
\begin{pmatrix}
    c_{1} \\
     \\
     \\
    \vdots \\
     \\
    c_{n}
\end{pmatrix}
$

$=
\begin{pmatrix}
    0 \\
    3(\frac{\Delta_{2}}{\delta_{2}} - \frac{\Delta_{1}}{\delta_{1}}) \\
     \\
    \vdots \\
     \\
     3(\frac{\Delta_{n-1}}{\delta_{n-1}} - \frac{\Delta_{n-2}}{\delta_{n-2}}) \\
    0
\end{pmatrix}
$

\vspace{10mm}
Στο αρχείο \textlatin{\textbf{splines.py}} υπάρχουν 4 συνάρτησεις. Η συνάρτηση 
\textlatin{\textit{splines\_interpolation}} δέχεται ως ορίσματα τα σημεία εκπαίδευσης και επιστρέφει τους συντελεστές της φυσικής κυβικής 
\textlatin{Splines} που προσεγγίζει τα σημεία αυτά, οι οποίοι υπολογίζονται σύμφωνα με αυτά που αναφέρθηκαν παραπάνω. Στην συνέχεια, η 
συνάρτηση \textlatin{\textit{find\_interval\_index}} επιστρέφει την θέση που θα έπρεπε να μπει το όρισμα \textlatin{search\_value} στον πίνακα
\textlatin{array}, ενώ η συνάρτηση \textlatin{\textit{calculate\_polynomial}} υπολογίζει την τιμή της φυσικής κυβικής \textlatin{Spline} στην 
θέση $x$ που δέχεται σαν παράμετρο. Τέλος, η συνάρτηση \textlatin{\textit{custom\_sin}} ορίζει αρχικά τα σημεία που αναφέρθηκαν στην παράγραφο
της εισαγωγής της \textit{Άσκησης 1} και στην συνέχεια χρησιμοποιεί τις άλλες τρεις συναρτήσεις για να προσεγγίσει την τιμή του ημιτόνου στο 
σημείο που δέχεται ως παράμετρο, αφού πρώτα το μεταφέρει στο διάστημα $[0,2\pi)$. 
\newpage
\end{document}