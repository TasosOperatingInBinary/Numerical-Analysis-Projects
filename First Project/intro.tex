\documentclass[First Project.tex]{subfiles}
\begin{document}

\section{ Εισαγωγή }

Η εργασία αυτή αναπτύχθηκε κατά την παρακολούθηση του μάθηματος της Αριθμητικής Ανάλυσης του 3ου εξαμήνου 
του τμήματος Πληροφορικής ΑΠΘ. Περιλαμβάνει υλοποιήσεις και πειραματισμούς με αλγορίθμους Αριθμητικής Ανάλυσης που αναφέρονται
σε θεματικές ενότητες όπως της επίλυσης μη γραμμικών εξισώσεων ( \textbf{Άσκηση 1 και 2} ) για παράδειγμα η μέθοδος της 
διχοτόμησης, της επίλυσης γραμμικών συστημάτων ( \textbf{Άσκηση 3} ) για παράδειγμα η μέθοδος της \textbf{
\textlatin{PA = LU}} αποσύνθεσης και τέλος της ιδιοανάλυσης ( \textbf{Άσκηση 4} ). Οι αλγόριθμοι αυτοί καθώς και οι
γραφικές παραστάσεις που περιέχονται στο παρόν έγγραφο έχουν υλοποιηθεί με την γλώσσα προγραμματισμού \textbf{\textlatin{Python}}
με χρήση των βιβλιοθηκών \textbf{\textlatin{NumPy}} και \textbf{\textlatin{Matplotlib}}. Στα παραδοτέα περιλαμβάνεται το παρόν 
\textbf{\textlatin{pdf}} αρχείο καθώς και το αντίστοιχο \textbf{\textlatin{.tex}} και για κάθε μία άσκηση υπάρχει ένας υποφάκελος 
όπου περίεχει τα αντίστοιχα \textbf{\textlatin{.py}} αρχεία με τον πηγαίο κώδικα.
\newpage

    
\end{document}