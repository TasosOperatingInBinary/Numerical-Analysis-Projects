\documentclass[First Project.tex]{subfiles}
\begin{document}

\subsection{Υπολογισμός σύγκλισης της μεθόδου \textlatin{\textbf{Newton-Raphson}} \\ για κάθε ρίζα}
    Για την μέθοδο \textlatin{\textbf{Newton-Raphson}} ζητείται επίσης να δειχθεί για ποιες ρίζες συγκλίνει τετραγωνικά και για ποιες όχι.
    Αρχικά αναφέρουμε τις τρεις ρίζες της συνάρτησης \textlatin{\textbf{f(x)}} με ακρίβεια 5ου ψηφίου.
    \vspace{10mm}
    \begin{figure}[h!]
        \centering
        \captionsetup{justification=centering}
        \begin{center}
            \begin{tabular}{ |c| }       
                \hline
                \textbf{$x_{0}$ = -1.1976} \\ \hline
                \textbf{$x_{1}$ = 1.5301} \\ \hline
                \textbf{$x_{2}$ = 0.0000} \\ 
                \hline
            \end{tabular}
            \caption{Ρίζες της συνάρτησης \textlatin{f(x)} στο διάστημα [-2,2]}
        \end{center}
    \end{figure}
    Στην συνέχεια αναφέρουμε το θεώρημα που θα χρησιμοποιηθεί για τον υπολογισμό της τάξης σύγκλισης της κάθε ρίζας.
    \large{\textbf{Θεώρημα : }} \normalsize Έστω ότι 
    \( x_{n} \neq x^{*} \)
    για κάθε φυσικό αριθμό \textlatin{n}, όπου $x^{*}$ σταθερό σημείο της συνάρτησης {\textbf{φ(\textlatin{x})}} που χρησιμοποιείται στην μέθοδο σταθερού 
    σημείου, και έστω ότι ισχύει :
    \begin{equation*}
        \lim_{n\to\infty} \frac{x_{n+1} - x^{*}}{(x_{n} - x^{*})^{p}} = a \neq  0 ,     
    \end{equation*}
    τότε η τάξη σύγκλισης της ακολουθίας $x_{n}$ είναι ακριβώς \textlatin{\textbf{p}}. \\ Πιο συγκεκριμένα για την μέθοδο 
    \textlatin{\textbf{Newton-Raphson}} χρησιμοποιώντας το θεώρημα \textlatin{\textbf{Taylor}} με κέντρο το σημείο \textlatin{\textbf{$x^{*}$}},
    όπου \textlatin{\textbf{$x^{*}$}} ρίζα της \textlatin{\textbf{f(x)}}, αποδεικνύεται ότι για την αναδρομική ακολουθία της μεθόδου 
    \textlatin{\textbf{Newton-Raphson}} το προηγούμενο όριο ισούται για \textlatin{\textbf{$p=2$}} με :
    \begin{equation*}
       \frac{f''(x^{*})}{2f'(x^{*})} \neq 0 ,
    \end{equation*}
    εφόσον \( f'(x^{*}) \neq 0\). \\
    Με αυτά που αναφέρθηκαν πλέον μπορούμε να προχωρήσουμε στον υπολογισμό της σύγκλισης της μεθόδου \textlatin{\textbf{Newton-Raphson}} για
    κάθε μία από τις τρεις ρίζες. \\ 
    
    Υπολογίζουμε αρχικά την πρώτη και την δεύτερη παράγωγο της \textlatin{\textbf{f(x)}} στο διάστημα \textbf{[-2.2]}:
    \begin{gather*}
            f'(x) = 3e^{sin^{3}x}*sin^{2}x*cosx + 6x^{5} - 8x^{3} -3x^{2} \\
            f''(x) = 9e^{sin^{3}x}*sin^{4}x*cos^{2}x + 3e^{sin^{3}x}*sin(2x)*cosx-3e^{sin^{3}x}*sin^{3}x + 30x^{4} - 24x^{2} - 6x            
    \end{gather*}
    
    \vspace{10mm}
    \begin{itemize}
        \item \textbf{$x_{0}$ = -1.1976} \\
            Ισχύει 
            \begin{equation*}
                f'( -1.1976 ) \approx 35.62888 \neq 0 ,
            \end{equation*}   
            επομένως μπορούμε να χρησιμοποιήσουμε τον προηγούμενο τύπο για την σύγκλιση της μεθόδου \textlatin{\textbf{Newton-Raphson}} για 
            την ρίζα \textlatin{\textbf{$x_{0}$}}. Οπότε έχουμε
            \begin{equation*}
                \lim_{n\to\infty} \frac{x_{n+1} - x_{0}}{(x_{n} - x_{0})^{2}} = \frac{f''(x_{0})}{2f'(x_{0})} = \frac{35.62888}{-2 * 4.92044} = -3.62049 \neq 0
            \end{equation*}
            επομένως η μέθοδος \textlatin{\textbf{Newton-Raphson}} συγκλίνει τετραγωνικά για την ρίζα \textbf{$x_{0}$ = -1.1976}.
        
        \item \textbf{$x_{1}$ = 1.5301} \\
            Ισχύει 
            \begin{equation*}
                f'( 1.5301 ) \approx 14.97241 \neq 0 ,
            \end{equation*}   
            επομένως μπορούμε να χρησιμοποιήσουμε τον προηγούμενο τύπο για την σύγκλιση της μεθόδου \textlatin{\textbf{Newton-Raphson}} για 
            την ρίζα \textlatin{\textbf{$x_{1}$}}. Οπότε έχουμε
            \begin{equation*}
                \lim_{n\to\infty} \frac{x_{n+1} - x_{1}}{(x_{n} - x_{1})^{2}} = \frac{f''(x_{1})}{2f'(x_{1})} = \frac{91.03090}{2 * 14.97241} = 3.03995 \neq 0
            \end{equation*}
            επομένως η μέθοδος \textlatin{\textbf{Newton-Raphson}} συγκλίνει τετραγωνικά για την ρίζα \textbf{$x_{1}$ = 1.5301}.
        \item \textbf{$x_{2}$ = 0.0000} \\
            Ισχύει 
            \begin{equation*}
                f'( 0.0000 ) = 0.0000 = 0 ,
            \end{equation*}   
            επομένως δεν μπορούμε να χρησιμοποιήσουμε τον προηγούμενο τύπο για την σύγκλιση της μεθόδου \textlatin{\textbf{Newton-Raphson}} για 
            την ρίζα \textlatin{\textbf{$x_{2}$}}. Αν υπολογίσουμε και την δεύτερη και τρίτη παράγωγο της \textlatin{\textbf{f(x)}} παρατηρούμε ότι
            κι αυτές έχουν ρίζα το \textbf{$x_{2}$ = 0.0000}, επομένως η ρίζα \textbf{$x_{2}$} έχει πολλαπλότητα \textbf{$m$ = 4}. Χρησιμοποιώντας το 
            ακόλουθο θεώρημα 
            
            \large{\textbf{Θεώρημα : }} \normalsize Έστω μία συνάρτηση \textlatin{\textbf{f(x)}} $( m+1 )$ φορές συνεχώς παραγωγίσιμη στο διάστημα \textlatin{\textbf{[a,b]}} κι
            έστω ότι η συνάρτηση έχει ως ρίζα το σημείο \textbf{\textlatin{$x$ = r}} με πολλαπλότητα \textbf{$m$}, τότε η μέθοδος \textlatin{\textbf{Newton-Raphson}}
            συγκλίνει τοπικά στο σημείο \textbf{\textlatin{$x$ = r}} και το σφάλμα \textbf{\textlatin{$e_{i}$}} στην \textlatin{\textbf{i}}-οστή  
            επανάληψη ικανοποιεί το παρακάτω 
            \begin{equation*}
                \lim_{i\to\infty} \frac{e_{i+1}}{e_{i}} = S,     
            \end{equation*}
            όπου
            \begin{equation*}
                S = \frac{m-1}{m}.
            \end{equation*}

            Χρησιμοποιώντας το παραπάνω θεώρημα έχουμε \(S = \frac {3}{4}\) οπότε η μέθοδος \textlatin{\textbf{Newton-Raphson}} συγκλίνει γραμμικά
            με \(e_{i+1} \approx \frac{3}{4} * e_{i}\). Επομένως η μέθοδος \textlatin{\textbf{Newton-Raphson}} δεν συγκλίνει τετραγωνικά λόγω της πολλαπλότητας της ρίζας.
            Αν η πολλαπλότητα μιας ρίζας \textlatin{\textbf{r}} είναι γνωστή από πριν μπορούμε να βελτιώσουμε την απόδοση της κλασσικής μεθόδου \textlatin{\textbf{Newton-Raphson}}
            με μία μικρή τροποποίηση :
            \begin{equation*}
                x_{n+1} = x_{n} - \frac{\textbf{\textlatin{$m * f(x_{n})$}}} { \textbf{\textlatin{$f'(x_{n})$}} }
            \end{equation*}
            όπου αποδεικνύεται ότι συγκλίνει τετραγωνικά στην ρίζα \textlatin{\textbf{r}}. Μία τέτοια συνάρτηση έχει υλοποιηθεί στο αρχείο \textit{\textlatin{\textbf{mult\_newton\_raphson.py}}}
            με όνομα \textit{\textlatin{\textbf{mult\_newton\_raphson}}}, όπου έχει προστεθεί ένα ακόμα όρισμα \textlatin{\textbf{m}} που δηλώνει
            την πολλαπλότητα της ρίζας που θέλουμε να υπολογιστεί από την συνάρτηση.
    \end{itemize}
\end{document}