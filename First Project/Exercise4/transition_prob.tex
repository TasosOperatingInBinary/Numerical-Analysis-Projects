\documentclass[First Project.tex]{subfiles}

\begin{document}
\subsection{ Πιθανότητα μεταπήδησης και αλλαγές στην τάξη σελίδας }

Σε αυτή την παράγραφο δοκίμαζονται νέες τιμές για την πιθανότητα μεταπήδησης, περιγράφεται ποιος είναι ο σκοπός της και οι αλλαγές στην τάξη
σελίδας. Η πιθανότητα μεταπήδησης $q$ μοντελοποιεί στον αλγόριθμο \textlatin{\textbf{PageRank}} την πιθανότητα ένας χρήστης να μετακινείται σε 
μία τυχαία σελίδα ενώ η τιμή $1-q$ μοντελοποιεί την πιθανότητα ο χρήστης να επιλέγει τυχαία έναν από τους συνδέσμους που βρίσκονται στην σελίδα
που βρίσκεται αυτή την στιγμή. Συγκεκριμένα, δοκίμαζονται οι τιμές $q=0.02$ και $q=0.6$ ως πιθανότητες μεταπήδησης.

\begin{figure}[h!]
    \centering
    \captionsetup{justification=centering}
    \begin{center}
        \begin{tabular}{ |c|c|c|c| }       
            \hline
            \textbf{\textlatin{$q=0.15$}} & \textbf{\textlatin{$q=0.02$}} & \textbf{\textlatin{$q=0.6$}} & Βαθμός εισόδου \\
            \hline 
            0.18983 & 0.21936 & 0.11171 & 4 \\
            0.10491 & 0.12544 & 0.07029 & 2 \\ 
            0.04993 & 0.04829 & 0.05960 & 2 \\
            0.02402 & 0.01223 & 0.05113 & 1 \\
            0.06734 & 0.07870 & 0.05890 & 2 \\
            0.05176 & 0.05343 & 0.05748 & 2 \\
            0.04857 & 0.04756 & 0.05709 & 2 \\
            0.03299 & 0.02229 & 0.05566 & 2 \\
            0.09754 & 0.11195 & 0.07148 & 2 \\
            0.13776 & 0.15074 & 0.10156 & 6 \\
            0.08137 & 0.06956 & 0.08744 & 5 \\
            0.03127 & 0.01639 & 0.05787 & 2 \\
            0.01591 & 0.00437 & 0.04568 & 1 \\
            0.02781 & 0.01245 & 0.05678 & 2 \\
            0.03898 & 0.02724 & 0.05734 & 2 \\
            \hline
        \end{tabular}
        \caption{ Τάξεις σελίδων για κάθε διαφορετική τιμή $q$ }
    \end{center}
\end{figure}

Τα συμπεράσματα που προκύπτουν αναλύοντας τον πίνακα του \textit{Σχήματος 52} είναι ότι η μείωση της πιθανότητας $q$ από $0.15$ σε $0.02$   
έχει ως αποτέλεσμα την μείωση της τάξης των σελίδων που έχουν χαμηλό βαθμό εισόδου καθώς με πιθανότητα $1-q = 1-0.02 = 0.98$, δηλαδή 98 στις
100 φορές, ο χρήστης επιλέγει μία σελίδα από αυτές που δείχνει η σελίδα που βρίσκεται, ενώ 2 στις 100 κατευθύνεται σε μία τυχαία από τις
διαθέσιμες σελίδες του δικτύου, οπότε οι τάξεις σελίδων με χαμηλό βαθμό εισόδου μειώνονται. Αντίστοιχα, η αύξηση της πιθανότητας $q$ από $0.15$ σε $0.6$ 
έχει ως αποτέλεσμα την μείωση της τάξης των σελίδων που έχουν υψηλό βαθμό εισόδου και τον διαμοιρασμό της σημαντικότητας τους στις σελίδες
με χαμηλότερο βαθμό εισόδου καθώς με πιθανότητα $1-q = 1-0.6 = 0.4$, δηλαδή 40 στις 100 φορές, ο χρήστης επιλέγει μία σελίδα από αυτές που
δείχνει	η σελίδα που βρίσκεται, ενώ 60 στις 100 κατευθύνεται σε μία τυχαία από τις διαθέσιμες σελίδες του δικτύου, με αποτέλεσμα οι τάξεις 
σελίδων με χαμηλό βαθμό εισόδου να αυξάνονται στις περισσότερες περιπτώσεις. Από τα παραπάνω καταλαβαίνουμε ότι ο σκοπός της πιθανότητας 
μεταπήδησης είναι η μοντελοποιήση της συμπεριφοράς ενός χρήστη που μετακινείται σε ένα δίκτυο. Βέβαια το παραπάνω μοντέλο, γνωστό και ως 
\textlatin{\textbf{Random Surfer Model}}, κάνει κάποιες παραδοχές που δεν ισχύουν στον πραγματικό κόσμο καθώς η κίνηση ενός χρήστη σε ένα 
δίκτυο δεν είναι τόσο απλή. Για παράδειγμα, στον πραγματικό κόσμο κανένας χρήστης δεν επιλέγει σε ποια σελίδα θα κατευθυνθεί με ίση 
πιθανότητα, ακόμα η τιμή της πιθανότητας $q$ είναι απλώς μία υπόθεση και τέλος δεν λαμβάνει υπόψιν τους σελιδοδείκτες ή το πλήκτρο της 
επιστροφής στην προηγούμενη σελίδα.

\end{document}