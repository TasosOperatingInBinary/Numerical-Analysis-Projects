\documentclass[First Project.tex]{subfiles}
\begin{document}


\section{ Άσκηση 4 }
Στην \textbf{4η άσκηση} πραγματοποιούνται πειραματισμοί με τον αλγόριθμο \textlatin{\textbf{PageRank}} που χρησιμοποιείται για την επιστροφή 
αποτελεσμάτων σε μηχανές αναζήτησης. Το υποθετικό δίκτυο αυτής της άσκησης αποτελείται από 15 ιστοσελίδες. Αρχικά, αποδεικνύεται ότι 
ο πίνακας \textlatin{Google} \textlatin{\textbf{G}}  είναι στοχαστικός. Στην συνέχεια, αποδεικνύεται ότι η μέγιστη ιδιοτιμή του παραπάνω πίνακα 
είναι $1$ και υπολογίζεται το ιδιοδιάνυσμα της μέγιστης ιδιοτιμής με την μέθοδο των δυνάμεων. Έπειτα,προσθέτονται 4 συνδέσεις και αφαιρείται 1 
από τον πίνακα γειτνίασης του δικτύου έτσι ώστε να βελτιωθεί ο βαθμός σημαντικότητας μιας σελίδας. Ακόμα, δοκίμαζονται νέες τιμές για την 
πιθανότητα μεταπήδησης, περιγράφεται ποιος είναι ο σκοπός της και οι αλλαγές στην τάξη σελίδας.Τέλος, γίνεται προσπάθεια βέλτιωσης της 
τάξης μίας σελίδας με μία διαφορετική στρατηγική και μελετάται η επίδραση της διαγραφής μίας συγκεκριμένης σελίδας από το δίκτυο στις τάξεις 
σελίδων.

\end{document}