\documentclass[First Project.tex]{subfiles}

\begin{document}
\subsection{ Απόδειξη Στοχαστικότητας πίνακα \textlatin{G}  }

Σε αυτήν την παράγραφο αποδεικνύεται αναλυτικά ότι ο πίνακας \textlatin{G} είναι στοχαστικός. 
Για να είναι ο πίνακας \textlatin{G} στοχαστικός πρέπει τα στοιχεία κάθε στήλης του να αθροίζουν στο 1. Τα στοιχεία του \textlatin{G} 
προκύπτουν από το παρακάτω 
\begin{equation*}
    G_{(i,j)} = \frac{q}{n} + \frac{A_{(j,i)}(1-q)}{n_{j}}, 
\end{equation*}
όπου
\begin{equation*}
    n_{j} = \sum_{i=1}^{n} A_{j,i}
\end{equation*}

Λαμβάνοντας υπόψιν τα παραπάνω θα πρέπει για κάθε στήλη $j$ του πίνακα \textlatin{G} να ισχύει 
\begin{equation*}
    \sum_{i=1}^{n} \frac{q}{n} + \frac{A_{(j,i)}(1-q)}{n_{j}} = 1
\end{equation*}

Οπότε
\begin{gather*}
    \sum_{i=1}^{n} \frac{q}{n} + \frac{A_{(j,i)}(1-q)}{n_{j}} = \sum_{i=1}^{n} \frac{q}{n} + \sum_{i=1}^{n} \frac{A_{(j,i)}(1-q)}{n_{j}} \\
    = \frac{q}{n} \sum_{i=1}^{n} 1 + \frac{1-q}{n_{j}} \sum_{i=1}^{n} A_{(j,i)} = \frac{q}{n} n +  \frac{1-q}{\sum_{i=1}^{n} A_{(j,i)}} \sum_{i=1}^{n} A_{(j,i)} \\
    = q + 1 - q = 1
\end{gather*}

όπου αποδεικνύεται ότι ο πίνακας \textlatin{G} είναι στοχαστικός αφού το άθροισμα κάθε στήλης $j$ κάνει 1.
\end{document}